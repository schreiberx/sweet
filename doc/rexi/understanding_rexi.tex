%% LyX 2.1.3 created this file.  For more info, see http://www.lyx.org/.
%% Do not edit unless you really know what you are doing.
\documentclass[english]{article}
\usepackage[T1]{fontenc}
\usepackage[latin9]{inputenc}
\usepackage{esint}
\usepackage{babel}
\begin{document}

\title{Understanding the rational approximation of the exponential integrator
(REXI)}

\maketitle
\begin{center}
\textbf{!!!THIS IS A PRELIMINARY, NON PROOF-READ DOCUMENT!!!}
\par\end{center}


\author{Martin Schreiber <M.Schreiber@exeter.ac.uk> et. al.}

This document serves as the basis for implementing the rational approximation
of the exponential integrator (REXI). Here, we purely focus on the
linear part of the shallow-water equations (SWE) and show the different
steps to approximate solving this linear part with an exponential
integrator. This paper mainly summarises previous work on REXI.


\section{Problem formulation}

We use the advective formulation of the SWE with a full linearization
with perturbation (see \cite{Formulations of the shallow-water equations})
with $U:=(h,u,v)^{T}$, yielding

\[
L(U(t)):=\left(\begin{array}{ccc}
 & H\delta_{x} & H\delta_{y}\\
g\delta_{x} &  & -f\\
g\delta_{y} & f
\end{array}\right)U(t)
\]


Here, we neglect all non-linear terms, assume only small perturbations
(hence negligible ones) around the average surface height and negligible
non-linear terms.

Then, the time evolution of the PDE with the subscript $t$ denoting
the derivative in time is given by

\[
U_{t}:=L(U).
\]


We continue writing the linear operator in matrix-style $L$ and applying
$L$ on $U$ as $L.U$.

\[
U_{t}:=L.U
\]


It is further worth noting, that this system describes an oscillatory
system, hence the operator $L$ has imaginary eigenvalues.


\section{Exponential integrator}

Linear systems of equations are well known to be solvable with exponential
integrators for arbitrary time step sizes via

\[
U(t):=e^{Lt}U(0),
\]
see e.g. \cite{Nineteen Dubious Ways to Compute the Exponential of a Matrix}.
However, this is typically quite expensive to compute and analytic
solutions only exist for some simplified system of equations, see
e.g. \cite{Formulations of the shallow-water equations} for f-plane
shallow-water equations. These exponential integrators can be approximated
with rational functions and this paper is on giving insight into this
approximation.


\section{Underlying idea of rational approximation}

Terry et. al. developed a rational approximation of the exponential
integrator, see \cite{High-order time-parallel approximation of evolution operators}.
First, we like to get more insight into it with a one-dimensional
formulation before applying REXI to a rational approximation of a
linear operator. Our main target is to find an approximation of an
operator with an \emph{exponential shape}, in our case $e^{ix}$,
which (in one-dimension) is given by a function $f(x)$. We will end
up in an approximation given by the following rational approximation:

\[
e^{ix}\approx\sum_{n=-N}^{N}\frac{\beta_{n}}{ix-\alpha_{n}}
\]
with complex coefficients $\alpha_{n}$ and $\beta_{n}$.


\subsection{Step A) Approximation of solution space}

First, we assume that we can use Gaussian curves as basis functions
for our approximation and they are somehow naturally related to this
problem (Gaussians are given by exponential functions). So first,
we find an approximation of one of our underlying Gaussian basis function

\[
\psi_{s}(x):=(4\pi)^{-\frac{1}{2}}e^{-x^{2}/(4s^{2})}
\]


In this formulation, $s$ can be interpreted as the horizontal ``stretching''
of the basis function. Note the similarities to the Gaussian distribution,
but by dropping certain parts of the vertical scaling as it is required
for probability distributions. We can now approximate our function
$f(x)$ with a superposition of basis functions $\psi_{s}(x)$ by

\[
f(x)\approx\sum_{m=-M}^{M}b_{m}\psi_{s}(x+ms)
\]
with $M$ controling the interval of approximation (\textasciitilde{}size
of ``domain of interest'') and $s$ the accuracy of integration
(\textasciitilde{}resolution in ``domain of interest''). The $s$
value depends on the max. frequency of the oscillations generated
by the linear operator. To compute the coefficients $b_{m}$, we rewrite
the previous equation in Fourier space with
\[
\frac{\hat{f}(\xi)}{\hat{\psi_{h}}(\xi)}:=\sum_{m:=\infty}^{\infty}b_{m}e^{2\pi imh\xi},
\]
see \cite{High-order time-parallel approximation of evolution operators},
page 11. We can rearrange this equation which allows us to directly
compute the desired coefficients 
\[
b_{m}:=h\intop_{-\frac{1}{2h}}^{\frac{1}{2h}}e^{-2\pi imh\xi}\frac{\hat{f}(\xi)}{\hat{\psi_{h}}(\xi)}d\xi.
\]
Since we are only interested in approximating $f(x):=e^{ix}$, we
can simplify the equation by using the response in frequency space
$\hat{f}(\xi):=\delta(\xi-\frac{1}{2\pi})$:

\[
b_{m}:=h\,e^{-2\pi imh\frac{1}{2\pi}}\frac{\hat{f}(\frac{1}{2\pi})}{\hat{\psi_{h}}(\frac{1}{2\pi})}=h\,e^{-imh}\hat{\psi_{h}}(\frac{1}{2\pi})^{-1}.
\]
We finally require to compute $\hat{\psi_{h}}(\xi)$ which we derive
in the following part:

\[
\hat{\psi_{h}}(\xi):=\intop_{-\infty}^{\infty}\frac{1}{\sqrt{4\pi}}e^{-\left(\frac{x}{2h}\right)^{2}}e^{-2\pi ix\xi}dx
\]
\[
=\frac{1}{\sqrt{4\pi}}\intop_{-\infty}^{\infty}e^{-\left(\left(\frac{x}{2h}\right)^{2}+2\pi ix\xi+(2h\pi i\xi)^{2}-(2h\pi i\xi)^{2}\right)}dx
\]
\[
=\frac{1}{\sqrt{4\pi}}e^{-(2h\pi\xi)^{2}}\intop_{-\infty}^{\infty}e^{-\left(\frac{x}{2h}+2h\pi i\xi\right)^{2}}dx
\]


\[
=\frac{1}{\sqrt{4\pi}}e^{-(2h\pi\xi)^{2}}\intop_{-\infty}^{\infty}e^{-\left(\frac{x}{2h}\right)^{2}}dx
\]
Then, evaluting the integral yields $\intop_{-\infty}^{\infty}e^{-\left(\frac{x}{2h}\right)^{2}}dx=h\,\sqrt{4\pi}$
and specialising on the case $\xi:=\frac{1}{2\pi}$, we get

\[
\hat{\psi_{h}\left(\xi\right)}:=\frac{1}{\sqrt{4\pi}}e^{-(2h\pi\frac{1}{2\pi})^{2}}h\sqrt{4\pi}=h\,e^{-h^{2}}.
\]
Finally, one can obtain the equation

\[
b_{m}:=h\,e^{-imh}\frac{1}{h\,e^{-h^{2}}}=e^{-imh}e^{h^{2}}
\]
to compute the coefficients $b_{m}$ for $f(x):=e^{ix}$.


\subsection{Step B) Approximation of basis function}

The second step is the approximation of the basis function $\psi_{s}(x)$
itself with a rational approximation, see \cite{Near optimal rational approximations of large data sets}.
Our basis function is given by

\[
\psi_{s}(x):=(4\pi)^{-\frac{1}{2}}e^{-x^{2}/(4s^{2})}
\]
and a close-to-optimal approximation of $\psi_{1}(x)$ with a sum
of rational functions is given by

\[
\psi_{1}(x)\approx Re\left(\sum_{l=-L}^{L}\frac{a_{l}}{ix+(\mu+i\,l)}\right)
\]
with the $\mu$ and $a_{l}$ given in \cite{Near optimal rational approximations of large data sets},
Table 1. We can generalise this approximation to arbitrary choosen
$s$ via 
\[
\psi_{s}(x)\approx Re\left(\sum_{l=-L}^{L}\frac{a_{l}}{i\frac{x}{s}+(\mu+i\,l)}\right)
\]



\subsection{Step C) Approximation of the approximation}

We then combine the approximation (B) of the approximation (A), yielding
\[
f(x)\approx\sum_{m=-M}^{M}c_{m}\psi_{s}(x+ms)=\sum_{m=-M}^{M}b_{m}Re\left(\sum_{l=-L}^{L}\frac{a_{l}}{i\frac{x+ms}{s}+(\mu+i\,l)}\right)
\]


\[
=\sum_{m=-M}^{M}b_{m}Re\left(\sum_{l=-L}^{L}\frac{a_{l}}{ix+s(\mu+i(m+l))}\right).
\]
We further like to simplify this equation and we observe, that for
$n:=n+l$, the denominator is equal. We can hence express parts of
the denomiator in terms of $n:=n+l$ by

\[
\alpha_{n}:=s(\mu+in).
\]
Now, we merge the $b_{m}$ and $a_{l}$ coefficients and first have
a look at the $b_{m}$ which is complex values. We observe the following
property: Assuming that we want to compute the real value of $f(x)$,
only the real value of $b_{m}$ has to be merged with the sum, since
the imaginary component would be dropped afterwards. This allows us
to move the $Re(b_{m})$ values inside the $\sum_{L}$:

\[
Re(f(x)):=Re\left(\sum_{m=-M}^{M}\,\,\sum_{l=-L}^{L}\frac{Re(b_{m})\,a_{l}}{ix+s(\mu+i(m+l))}\right).
\]
Now we can collect all nominators with equivalent denominator (if
$n=m+l$ and by using $\delta$ as the Kronecker delta), yielding

\[
\beta_{n}^{Re}:=\sum_{m=-M\,}^{M}\sum_{l=-L}^{L}Re(b_{m})a_{l}\delta(n,\,m+l)
\]
for real values $f(x)$ and

\[
\beta_{n}^{Im}:=\sum_{m=-M\,}^{M}\sum_{l=-L}^{L}Im(b_{m})a_{l}\delta(n,\,m+l)
\]
for complex values of $f(x)$. This finally leads us to the REXI approximation

\[
e^{ix}\approx\sum_{n=-N}^{N}Re\left(\frac{\beta_{n}^{Re}}{ix+\alpha_{n}}\right)+i\,Re\left(\frac{\beta_{n}^{Im}}{ix+\alpha_{n}}\right)
\]
for the complex-valued function $e^{ix}$.


\section{Apply REXI with a matrix:}

Finally, we like to apply REXI to a formulation such as 
\[
U(t):=e^{tL}U(0).
\]
To see the relationship between the approximation of $e^{ix}$ with
REXI, we first rewrite the exponential formulation in terms of $f(x):=e^{itx}$
\[
f(L):=e^{tL}=\Sigma\Lambda\Sigma^{H}=\Sigma\left(\begin{array}{ccc}
...\\
 & e^{i\lambda_{n}t}\\
 &  & ...
\end{array}\right)\Sigma^{H}=\Sigma\,f(\lambda_{n})\,\Sigma^{H}
\]
with complex-valued exponentials on the eigenvalues. Hence, the accuracy
of the exponential integrator on $f(L)$ only depends on the spectrum
of the $L$ and allows to be applied in the same way as $e^{ix}$,
but by replacing $x$ with the matrix $L$. For error bounds, we like
to refer to \cite{High-order time-parallel approximation of evolution operators}.


\section{Filtering}

Since we are approximating the exponential function with a series,
one of the most important properties can be violated: The evaluation
of $e^{ix}$ is bounded by unity (think of it as a series of real-valued
$cos$ and imaginary-valued $sin$).

However, the interpolated values can exceed this unity due to interpolation
properties (think of a Lagrangian interpolation of high order, leading
to large oscillations with the possibility of exceeding the local
min/max of the interpolated function in the area of support). This
can lead to long-term effects such as amplifying unphysical solutions.
Therefore a filtering may be required to assure that the function
in the interpolation range is always bounded by unity.


\section{REXI, our little dog}

In the following, we use $L:=\tau L'$ and assume an a-priori fixed
time step size, making a REXI approximation more efficient. Then,
the REXI approximation is given by

\begin{equation}
exp(\tau L')\approx\sum_{k=-K}^{K}\beta_{k}(L-\alpha_{k})^{-1}\label{eq:rexi}
\end{equation}


The coefficients $\alpha_{k}$ (corresponding to $s(\mu+i\,n)$ in
step C for the one-dimensional formulation) can be precomputed or
computed during program start. $\mu$ is based on a one-dimensional
approximation, see the paper, and the $\alpha_{k}$ can be interpreted
as shifts of the rational approximations. The coefficients $\beta_{k}$
(corresponding to $c_{n}$ in step C) are describing the scaling of
the basis function and are also constant and independent of the solution
itself. Note, that for debugging purpose, their \emph{imaginary values
have to cancel out}.

Note an important property (see Sec. 3.3 in \cite{High-order time-parallel approximation of evolution operators}).
There's an anti-symmetry in the $\alpha_{i}$ coefficients, which
avoids computing half of the inverses:

\[
\overline{(L-\alpha)^{-1}U(0)}=(L-\overline{\alpha})^{-1}U(0)
\]



\section{Computing inverse of $(L-\alpha)^{-1}$}

For computing the inverse, arbitrary solvers can be used. However
we like to note, that $\alpha$ is a complex number. Hence, requiring
solvers with support for solving in complex space. As an example,
we consider a specialization on the shallow-water equations given
above with

\[
L(U(t)):=\left(\begin{array}{ccc}
 & H\delta_{x} & H\delta_{y}\\
g\delta_{x} &  & -f\\
g\delta_{y} & f
\end{array}\right)U(t)
\]
\[
U_{t}(t):=L(U(t))
\]
and we set $g:=1$ and the average height $H:=1$. 


\subsection{Handling $\tau$ in REXI\label{sub:Handling-tau-in-REXI}}

We recall the formulation of the solution as an exponential integrator

\[
U(t):=e^{tL}U(0)
\]
which formally allows us to join the integration in time givey by
$t$ with the $L$ operator in case of such a formulation. There are
basically two different ways to handle this scaling:

The first one is rescaling all parameters by $\tau$:

\[
g':=\tau g,\,\,\,\,\,f':=\tau f,\,\,\,\,\,h_{0}':=\tau h_{0}.
\]


The second way is to reformulate the REXI approximation scheme given
by

\[
(\tau L-\alpha)^{-1}.U(\tau)=U(0)
\]
and by factoring $\tau$ out, yielding

\[
(L-\frac{\alpha}{\tau})^{-1}.U(\tau)\tau^{-1}=U(0)
\]
So instead of solving for $U(\tau)$, we are solving for $U^{\tau}(\tau):=U(\tau)\tau^{-1}$
as well as $\alpha^{\tau}:=\frac{\alpha}{\tau}$.

To summarize, we have to solve the system of equations given by

\begin{equation}
(L-\alpha^{\tau})^{-1}.U^{\tau}(1)=U(0)\label{eq:unit_rexi_timestep}
\end{equation}
with $U(0)$ the initial conditions, $\alpha^{\tau}:=\frac{\alpha}{\tau}$
and $U(\tau):=U^{\tau}(1)\tau$. For sake of simplicity, we stick
to the formulation without the prime notation.

One final scaling has to be done: the exponential is computing $e^{\tau L}$,
hence the real-valued $\tau$ has to be included in the operator $L$.
There are basically two different ways: The first one is rescaling
all parameters by $\tau$:

\[
g^{\tau}:=\tau g,\,\,\,\,\,f^{\tau}:=\tau f,\,\,\,\,\,h_{0}^{\tau}:=\tau h_{0}
\]
The second way is to factor the $\tau$ parameter out:

\[
\left(\tau L-\alpha\right)^{-1}.U(\tau)=U(0)
\]


\[
\left(L-\frac{\alpha}{\tau}\right)^{-1}.U(\tau)\tau^{-1}=U(0)
\]
So instead of solving for $U(\tau)$, we are solving for $U^{\tau}(\tau):=U(\tau)\tau$
as well as $\alpha^{\tau}:=\frac{\alpha}{\tau}$and have to divide
the computed solution by $\tau$ in the end.


\subsection{Solving as an eliptic problem}

{[}TODO (Pedro): Derive dimensional formulation{]}

We like to mention again, that we can use arbitrary solvers and in
this work, we focus on a reformulation into an eliptic problem. Following
the idea in \cite{An invariant theory of the linearized shallow water equations with rotation and its application to a sphere and a plane},
instead of solving this relatively large system of equations we can
split the problem into an elliptic one for the height which then allows
to use an explicit formulation for the velocities. We use the abbreviation$\vec{v}:=(u,v)$
in the following paragraph and the formulation with a unit time step
(see Eq. \ref{eq:unit_rexi_timestep}). Using the formulation in \cite{High-order time-parallel approximation of evolution operators},
the height can be computed with the elliptic equation given by

\begin{equation}
(\nabla^{2}-(\alpha^{2}+f^{2}))h(1)=\frac{\alpha^{2}+f^{2}}{\alpha}(h(0)+H\nabla\cdot(A\,\vec{v}(0))\label{eq:elliptic_height}
\end{equation}
with

\[
A:=\frac{1}{\alpha^{2}+f^{2}}\left(\begin{array}{cc}
\alpha & -f\\
f & \alpha
\end{array}\right).
\]



\subsection{f-plane}

Assuming an f-plane approximation (f is constant), we can rearrange
this equation by using the abbreviations $\kappa:=\alpha^{2}+f^{2}$
in the following way:

\[
(\nabla^{2}-\kappa)\,h(1)=\frac{\kappa}{\alpha}\left(h(0)+\nabla\cdot(A\,\vec{v}(0)\right)
\]


\[
(\nabla^{2}-\kappa)\,h(1)=\frac{\kappa}{\alpha}h(0)+\frac{1}{\alpha}\nabla\cdot\left(\begin{array}{cc}
\alpha & -f\\
f & \alpha
\end{array}\right)\vec{v}(0)
\]


\begin{equation}
(\nabla^{2}-\kappa)h(1)=\frac{\kappa}{\alpha}h(0)-\frac{f}{\alpha}\nabla\times v(0)+\nabla\cdot\vec{v}(0)
\end{equation}
Here, the $\alpha$ and $\kappa$ denote the terms with imaginary
numbers and this formulation should also simplify programming.

We continue with an interpretation of this formulation: on the right
hand side we see an update-like scheme $h(0)$ in the first scheme,
then a vorticity-like formulation $\times$, and an advective part
$\nabla$. To simplify the notation for solving the system, we rewrite
it as

\begin{equation}
(\nabla^{2}-\kappa)\,h(1)=D
\end{equation}
with real-and-imaginary-valued $D$ and $\nabla^{2}-\kappa$ as well
as a real-and-imaginary-valued $h(1)$ for which we want to solve.
Then, the solution is given e.g. in spectral space directly via
\[
h(1):=D(\nabla^{2}-\kappa)^{-1}
\]
Once computed the height, the velocities can be directly computed
via

\begin{equation}
\vec{v}(1)=-A.\vec{v}(0)+A.\nabla h(1)=-A.(\vec{v}(0)-\nabla h(1))\label{eq:elliptic_velocity}
\end{equation}
giving us our final solution

\[
U(\tau):=\tau(h(1),\,u(1),\,v(1))^{T}
\]
with the scaling with $\tau$ as discussed in Sec. \ref{sub:Handling-tau-in-REXI}
and we like to mention, that also $\alpha$ has to be scaled appropriately
before using it for REXI.


\subsection{Interpretation of $\tau$}

We like to close this section with a brief discussion of $\tau$ by
having a look on the REXI reformulation

\[
(L-\frac{\alpha}{\tau})^{-1}.U(\tau)\tau=U(0)
\]
We see, that for an increasing $\tau$, hence an integration in time
over a larger time period, the poles given by $\alpha$ are getting
closer. This can possibly lead to a loss in accuracy for the data
sampled by the outer poles $\alpha_{-N}$ and $\alpha_{N}$. Therefore,
the number $N$ of poles is expected to scale linearly with the size
of the coarse time step:
\[
|N|\sim\tau
\]


{[}TODO (Terry): There's probably a tighter relationship somewhere
hidden in the paper{]}


\section{Bringing everything together}

Using the spectral methods (e.g. in SWEET), we can directly solve
the height Eq. (\ref{eq:elliptic_height}) and then solver for the
velocity in Eq. (\ref{eq:elliptic_velocity}). Then, the problem is
reduced to computing the REXI as given in Eq. (\ref{eq:rexi}). We
like to note again, that the $\alpha_{n}$ and $\beta_{n}$ are independent
of the system $L$ to solve, and the number of coefficients only depends
on the accuracy and the resolution.


\section{Notes on HPC}
\begin{itemize}
\item The terms in REXI to solve are all independent. Hence, for latency
avoiding, the communication can be interleaved with computations.
\item The iterative solvers are memory bound. Instead of computing $c:=a*b$
for the stencil operations, we could compute $\vec{c}:=a\vec{b}$
with $a$ one coefficient in the stencil. This allows vectorization
over $c$ and $b$ on accelerator cards with strided memory access.
\item It is unknown which method is more efficient to solve the system of
equations:

\begin{itemize}
\item iterative solvers have low memory access,
\item inverting the system and storing it as a sparse matrix allows fast
direct solving but can yield more memory access operations.
\end{itemize}
\item Splitting the solver into real and complex number would store them
consecutively in memory. This has a potential to avoid non-strided
memory access and using the same SIMD operations (Just a rough idea,
TODO: check if this is really the case).
\end{itemize}

\section{Acknowledgements}

Thanks to Pedro \& Terry for the feedback \& discussions!
\begin{thebibliography}{1}
\bibitem{Formulations of the shallow-water equations}Formulations
of the shallow-water equations, M. Schreiber

\bibitem{High-order time-parallel approximation of evolution operators}High-order
time-parallel approximation of evolution operators, T. Haut et. al.

\bibitem{An asymptotic parallel-in-time method for highly oscillatory PDEs}An
asymptotic parallel-in-time method for highly oscillatory PDEs, T.
Haut et. al.

\bibitem{An invariant theory of the linearized shallow water equations with rotation and its application to a sphere and a plane}An
invariant theory of the linearized shallow water equations with rotation
and its application to a sphere and a plane, N. Paldor et. al.

\bibitem{Nineteen Dubious Ways to Compute the Exponential of a Matrix}Nineteen
Dubious Ways to Compute the Exponential of a Matrix, Twenty-Five Years
Later, Cleve Moler and Charles Van Loan, SIAM review

\bibitem{Near optimal rational approximations of large data sets}Near
optimal rational approximations of large data sets, Damle, A., Beylkin,
G., Haut, T. S. \& Monzon\end{thebibliography}

\end{document}
