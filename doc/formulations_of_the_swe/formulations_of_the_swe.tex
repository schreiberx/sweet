%% LyX 2.1.4 created this file.  For more info, see http://www.lyx.org/.
%% Do not edit unless you really know what you are doing.
\documentclass[english]{article}
\usepackage[T1]{fontenc}
\usepackage[latin9]{inputenc}

\makeatletter

%%%%%%%%%%%%%%%%%%%%%%%%%%%%%% LyX specific LaTeX commands.
%% Because html converters don't know tabularnewline
\providecommand{\tabularnewline}{\\}

%%%%%%%%%%%%%%%%%%%%%%%%%%%%%% User specified LaTeX commands.
\usepackage{babel}

\makeatother

\usepackage{babel}
\begin{document}

\title{Formulations of the shallow-water equations}


\author{Martin Schreiber <M.Schreiber@exeter.ac.uk>\\
 Pedro S. Peixoto <pedrosp@ime.usp.br>}

\maketitle
There are a variety of different formulations for the shallow-water
equations available. This document serves as an overview of these
different formulations and how they are approximated.


\section{Overview}

We will adopt the following terminology.\\
 %
\begin{tabular}{|c|l|}
\hline 
\textbf{Symbol}  & \textbf{Description}\tabularnewline
\hline 
\hline 
$\eta$  & Fluid layer depth/height\tabularnewline
\hline 
$\bar{\eta}$  & Mean water depth\tabularnewline
\hline 
$\vec{u}$  & 2D velocity \tabularnewline
\hline 
$u$  & Velocity in x direction\tabularnewline
\hline 
$v$  & Velocity in y direction\tabularnewline
\hline 
$\eta u$  & Momentum in x direction\tabularnewline
\hline 
$\eta v$  & Momentum in y direction\tabularnewline
\hline 
$U$  & Vector with $(\eta,u,v)$\tabularnewline
\hline 
$\zeta_{R}$  & Relative vorticity\tabularnewline
\hline 
$\zeta_{A}$  & Absolute vorticity ($\zeta_{R}+f$)\tabularnewline
\hline 
$q$  & Potential vorticity ($\frac{\zeta_{A}}{\eta}$)\tabularnewline
\hline 
$g$  & Gravity\tabularnewline
\hline 
$\Phi$  & Bernoulli potential\tabularnewline
\hline 
$f$  & Coriolis force $2\Omega sin(\phi)$\tabularnewline
\hline 
$\vec{k}$  & Vector pointing outward of the plane\tabularnewline
\hline 
\end{tabular}\\


We will also assume no bottom topography (orography/bathymetry), so
the fluid layer height $\eta$ is the total height. Most of what is
described here is as well described in \cite{Williamson1992}.


\section{Flux formulation (momentum, conservative formulation)}

The conservative formulation uses the momentum $(\eta u,\,\eta v)$
as the conserved quantities. In this form, the equations are given
as

\[
\eta_{t}+(\eta u)_{x}+(\eta v)_{y}=0
\]


\[
(\eta u)_{t}+(\eta u^{2}+\frac{1}{2}g\eta^{2})_{x}+(\eta uv)_{y}-fv=0
\]


\[
(\eta v)_{t}+(\eta v^{2}+\frac{1}{2}g\eta^{2})_{y}+(\eta uv)_{x}+fu=0
\]
Since this formulation is conserving the momentum, it is also referred
as the conservative form.

In vector notation we have it as, 
\[
\eta_{t}+\nabla\cdot(\eta\vec{u})=0
\]
\[
(\eta\vec{u})_{t}+\nabla\cdot(\vec{u}\eta\vec{u})+f\vec{k}\times(\eta\vec{v})+gh\nabla\eta=0
\]
Please note, that this formulation contains several non-linear terms.


\section{Advective formulation (non-conservative formulation)}

This is a formulation which makes it more challenging to get conservative
properties such as for the momentum, hence it is also called non-conservative
formulation. By using the produce rule, we can reformulate the equations
above to make the velocities conserved quantities. Here, we first
apply the produce rule to $(\eta u)_{t}$

\[
\eta u_{t}+\eta_{t}u+\eta_{x}u^{2}+2\eta uu_{x}+g\eta\eta_{x}+(\eta u)_{y}v+\eta uv_{y}-fv=0
\]


\[
\eta u_{t}+(-(\eta u)_{x}-(\eta v)_{y})u+\eta_{x}u^{2}+2\eta uu_{x}+g\eta\eta_{x}+(\eta v)_{y}u+\eta vu_{y}-fv=0
\]


\[
\eta u_{t}+(-(\eta u)_{x})u+\eta_{x}u^{2}+2\eta uu_{x}+g\eta\eta_{x}+\eta vu_{y}-fv=0
\]


\[
\eta u_{t}-\eta_{x}uu-\eta uu{}_{x}+\eta_{x}u^{2}+2\eta uu_{x}+g\eta\eta_{x}+\eta vu_{y}-fv=0
\]


\[
u_{t}+g\eta_{x}+uu_{x}+vu_{y}-fv=0
\]
Altogether, we get the following non-conservative formulation:

\[
\eta_{t}+u\eta_{x}+v\eta_{y}+\eta u_{x}+\eta v_{y}=0
\]
\[
u_{t}+uu_{x}+vu_{y}-fv+g\eta_{x}=0
\]
\[
v_{t}+uv_{x}+vv_{y}+fu+g\eta_{y}=0
\]
We can also split these equations in linear and non-linear parts
\[
U_{t}=L(U)+N(U)
\]


\[
L(U):=\left(\begin{array}{ccc}
\\
-g\partial_{x} &  & f\\
-g\partial_{y} & -f
\end{array}\right)U
\]


\[
N(U):=\left(\begin{array}{c}
-(\eta u)_{x}-(\eta v)_{y}\\
-uu_{x}-vu_{y}\\
-uv_{x}-vv_{y}
\end{array}\right)
\]
with $U:=(\eta,u,v)^{T}$. Defining the material derivative as 
\[
\frac{d}{dt}:=\frac{\partial}{\partial t}+\vec{u}\cdot\nabla
\]
we may write the equations in vector notation, 
\[
\frac{d\eta}{dt}+\eta\nabla\cdot\vec{u}=0,
\]


\[
\frac{d\vec{u}}{dt}+f\vec{k}\times\vec{v}+g\nabla\eta=0.
\]



\section{Vector invariant formulation (vorticity formulation)}

The vector invariant formulation is used in models such as NEMO, GungHo,
MPAS, etc. The discretization of it strongly differs between the different
models.

For certain reasons such as energy and enstropy conservation, the
equations are reformulated and split into the vorticity and the Bernoulli
potential $\Phi$ component, defined as 
\[
\Phi:=g\eta+\frac{u^{2}+v^{2}}{2}=g\eta+\frac{\vec{u}\cdot\vec{u}}{2}.
\]


We first add and subtract an artificial term $vv_{x}$

\[
u_{t}-vv_{x}+g\eta_{x}+uu_{x}+vv_{x}+vu_{y}-fv=0,
\]
which leads to,

\[
u_{t}-vv_{x}+vu_{y}+(g\eta+\frac{u^{2}+v^{2}}{2})_{x}-fv=0,
\]
with $\zeta_{A}:=v_{x}-u_{y}+f$, yielding

\[
u_{t}-v\zeta_{A}+\Phi_{x}=0.
\]
Similarly, we can reformulate $v_{t}$,

\[
v_{t}+g\eta_{y}+uv_{x}+vv_{y}+uu_{y}-uu_{y}+fu=0
\]


\[
v_{t}+uv_{x}-uu_{y}+fu+(g\eta+\frac{v^{2}+u^{2}}{2})=0
\]
This yields the vorticity formulation,

\[
\eta_{t}+(\eta u)_{x}+(\eta v)_{y}=0,
\]


\[
u_{t}-v\zeta_{A}+\Phi_{x}=0,
\]


\[
v_{t}+u\zeta_{A}+\Phi_{y}=0.
\]
In vector notation we have 
\[
\eta_{t}+\nabla\cdot(\eta\vec{u})=0,
\]


\[
\vec{u}_{t}+\zeta_{A}\vec{k}\times\vec{u}+\nabla\Phi=0.
\]



\section{Fluid depth perturbation}

It is convenient to split the fluid depth into a mean height and a
perturbation, such that we express $\eta=\bar{\eta}+\eta'$, with
$\bar{\eta}$ being the assumed constant mean water depth. Since the
gradient of $\eta$ will be the same as the gradient of $\eta'$,
the only change happens in the continuity equation, 
\[
\eta_{t}+\nabla\cdot((\bar{\eta}+\eta')\vec{u})=0,
\]
\[
\eta'_{t}=-\bar{\eta}\nabla\cdot\vec{u}-\nabla\cdot(\eta'\vec{u})
\]
and the momentum equations can be adopted with your preferred formulation
but now considering $\eta'$ instead of $\eta$.

We can write these equations, in any of the above formulations, as
a linear operator and a non linear one

\[
L(U):=\left(\begin{array}{ccc}
0 & -\bar{\eta}\partial_{x} & -\bar{\eta}\partial_{y}\\
-g\partial_{x} & 0 & f\\
-g\partial_{y} & -f & 0
\end{array}\right)U,
\]


\[
U_{t}=L(U)+N(U)
\]
where the nonlinear part $N(U)$ will depend on the equation formulation.
\begin{thebibliography}{1}
\bibitem{Williamson1992} Williamson, David L. and Drake, John B.
and Hack, James J. and Jakob, R�diger and Swarztrauber, Paul N. A
standard test set for numerical approximations to the shallow water
equations in spherical geometry, J. Comput. Phys, 1992. \end{thebibliography}

\end{document}
